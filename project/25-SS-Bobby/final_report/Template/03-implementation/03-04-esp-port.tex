%-----------------------------------------------------------------------
\section{Dancing Bear ESP32-Port}
\label{sec:impl:esp-port}
%-----------------------------------------------------------------------
One of the initial ideas of the project was running the Dancing Bear signature scheme on multiple ESP32 microcontrollers and if possible on smart-cards.
However, the latter has not been progressed as the focus was shifted towards a functioning PC implementation and the ESP32-port.

The port is contained in the \textit{esp\_dancing\_bear} folder and has to be set up as an ESP-IDF project.
Setup and building instructions are described in \textit{README.md} of the previously mentioned folder.
Therefore, those instructions will not be part of this documentation.
Nevertheless, it is important to mention that the IDF version v4.4.2 is mandatory.
Version v5.x of the \texttt{ESPRESSIF-IDF} changed the inclusion of different libraries, which leads to compile errors of the \texttt{Asio} library.
Not including ``asio.hpp`` as the first library may also lead to compile errors, thus including it as the first is recommended.

Variations from the PC-version of Dancing Bear described in section~\ref{sec:impl:dancing-bear} can be found in the usage of \texttt{Asio} or more specifically \texttt{Boost.Asio} plus ESP-specific code for the internet connection of the ESPs.
The slightly more sophisticated library \texttt{Boost.Asio} does not run on the ESP32, only the simplistic version \texttt{Asio} is supported.
Hence, \texttt{Boost} calls had to be replaced with mostly \texttt{Asio} and sometimes different standard library calls.
A notable change had to be done in the function \texttt{async\_accept()} because the ESP version of \texttt{Asio} does not support the \texttt{emplace} function of a socket.

Furthermore, the compiler only supports C++14 and thus the C++17 standard library ``optional`` had to be removed completely.
The ESP-port implements solely the trustee side of the protocol.
Neither dealer, helper nor verifier are intended to be run on an ESP and therefore not supported by this port.

\subsection*{Running the Scheme and Known Problems}\label{subsec:running-the-scheme-and-known-problems}
The code can be built and flashed to a single ESP32 after setting up the required data as described in the instructions of the \textit{README.md}.
Before building and flashing to another device, the device IP should be replaced.
The protocol can be executed by a normal computer when all needed ESP32-devices are prepared.
Firstly, a helper should be started and kept running.
Secondly, the Dealer has to arrange the needed data and send them to the corresponding participants.
Lastly, when all data has been distributed a signing request can be proposed.
See~\ref{sec:impl:dancing-bear} for an in-depth description since all the steps are the same except for the replacement of the trustees on multiple computers or terminals with EPS32-devices.
A combination of computers and EPS32-devices is also possible and may be required.

Currently, when running the scheme on ESP32-devices it will fail after successfully signing a single signature.
During the implementation of the ESP32-port, multiple problems concerning the storage of the devices have been encountered.
Firstly, the flash-size is too large for the default partition table.
Hence, a modified partition table has to be used to successfully flash the code to the devices.
The \textit{README.md} explains how to use the correct partition table.
Secondly, flashing has also been prevented by not having enough allocated space overall.
Thus, a flash size of at least 4MB must be enforced.

After solving those problems, the code runs fine until a signing request is made.
When a signature is being requested the EPS32 accepts the request, calculates its share and answers.
Instead of behaving normally, and therefore going back to its idle state, the device will encounter a non-specified critical error and reboots resulting in a loss of all data sent by the dealer.
Following extensive research, it is presumed to be also an issue related to missing memory, but in connection with the used \texttt{JSON} library.
Furthermore, the ESP32 will also crash and reboot when receiving the request to initiate a signature.
Therefore, running an ESP32 as the initiating trustee is currently not possible at all and a proper computer has to be used to get a single valid signature.
It has not been possible to resolve those difficulties during this project, but they have to be tackled in future work.

