%-----------------------------------------------------------------------
\section{Third-Party Libraries}
\label{sec:impl:libs}
%-----------------------------------------------------------------------

Third-party libraries were used to implementation the Dancing Bear signature scheme.
In the following section all of them are discussed.

For the generation of \ac{JSON} strings and their parsing, the \texttt{nlohmann-json}~\cite{nlohmannjson} library was used.
It was chosen because it is well documented, compatible with all \texttt{C++} \ac{STL} containers, and offers a simple yet feature-rich interface.

The \texttt{Asio}~\cite{asio} network communication library was used due to its compatibility with the ESP32 system and its asynchronous working principle.
This splits the network communication and the handling of the messages to different threads, enabling efficient parallelization.

Additionally, the \texttt{Boost.program\_options}~\cite{boostpo} library was used to parse command line arguments for simplicity.

Moreover, \texttt{merklecpp}~\cite{merklecpp} is used for Merkle trees.
Its functionality to serialize and deserialize single nodes, verification paths, and even whole trees were proven to be extremely useful for the present implementation.
However, it was later discovered to be unsuitable in specific instances and was therefore modified.
The modifications included adding support for more hash functions and changing the verification process of a Merkle path object.
In the original version, a path for a specified leaf is obtained by calling the \texttt{path} method with a given leaf number on a tree.
The resulting path includes no method for setting the leaf element, which was necessary to directly use the included verify method.
A method for setting the leaf of a Merkle path was added for this reason.

Lastly, we took the implementation of the \texttt{SHAKE} hash functions from the \ac{XMSS} reference code~\cite{fips}.


