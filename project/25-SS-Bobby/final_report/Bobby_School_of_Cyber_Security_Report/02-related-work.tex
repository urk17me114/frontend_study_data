%-----------------------------------------------------------------------
\section{Related Work}
\label{sec:related-work}
%----------------------------------------------------------------

The foundation of this project draws heavily from the \textbf{OWASP Juice Shop}, 
an intentionally vulnerable web application developed and maintained by the Open Web Application Security Project (OWASP). 
Juice Shop is widely recognized as one of the most comprehensive training platforms for web application security, 
as it incorporates vulnerabilities that map directly to the \textbf{OWASP Top 10} categories, 
as well as numerous other security flaws commonly encountered in real-world systems.  
It has been widely used in security education and training, offering a realistic, hands-on environment for us to explore and understand the impact of vulnerabilities.

In this project, the Juice Shop served not only as a reference but also as a practical learning tool. 
We were explicitly instructed by the course instructors to read each challenge, study the provided hints, and attempt exploitation without consulting the official solutions. 
This pedagogical constraint required learners to invest significant time in trial-and-error, experimentation, and collaborative problem-solving. 
Through this process, participants developed a deeper conceptual understanding of the vulnerability classes they encountered (for example, sql injection, IDOR, XXE, and security misconfiguration), and gained practical skills in reconnaissance, exploitation techniques, and evidence collection. 
The deliberate emphasis on working through challenges from hints fostered perseverance, improved debugging skills, and a stronger mental model of how real-world attackers probe and exploit application weaknesses.

Through this engagement, we encountered a wide range of vulnerabilities, including \textit{SQL injection}, 
\textit{broken authentication}, 
\textit{insecure direct object references (IDOR)}, 
\textit{security misconfigurations}, 
and \textit{XML External Entity (XXE) attacks}. 
These exercises provided both theoretical grounding in web security concepts and practical experience in identifying and mitigating risks, bridging the gap between abstract security principles and real-world application.

Beyond the Juice Shop, prior research emphasizes the importance of staged or multi-level educational environments for learning secure software development. 
Studies have shown that learners gain deeper comprehension when exposed first to vulnerable systems, followed by guided mitigation exercises that illustrate the effect of security improvements. 
In line with this, the project implements three distinct configurations: 
\textbf{System 1}, the vulnerable system, serves as a baseline for observing exploitation; 
\textbf{System 2} and \textbf{System 3}, the partially secured system, demonstrates the impact of resolving selected vulnerabilities; 
This staged approach allows us to directly observe the effect of security interventions on system behavior and resilience, reinforcing lessons about secure coding and access control.

The project also aligns with other educational initiatives and platforms that combine vulnerability exploration with remediation, such as WebGoat and Hackademic, while contributing a distinct emphasis on comparative analysis across multiple system versions. 
By intentionally implementing and then progressively hardening the same application, the \textbf{Bobby School of Cyber Security} enables learners to observe not only how vulnerabilities are exploited, but also how targeted and layered mitigations alter the attack surface and system behavior.

By incorporating the OWASP Juice Shop methodology and enforcing a challenge-driven learning model, this project contributes to security education by combining hands-on exploitation, focused mitigation development, and reflective analysis. 
The approach emphasizes both technical skill acquisition and the development of security reasoning — preparing us to apply secure software engineering practices in real-world development and assessment contexts.
