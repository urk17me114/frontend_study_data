\section{Conclusion}

The cyber security project aimed to design and implement a School Management System capable of demonstrating both the existence and mitigation of software vulnerabilities within a realistic web application environment. The system was developed with three primary user roles - administrator, teacher, and student and incorporated essential academic management functionalities such as registration and authentication, announcement publishing, timetable generation, assignment upload and evaluation, and marks management. These features collectively represented a functional, database-driven application suitable for security testing and configuration analysis.

In the initial phase, a deliberately vulnerable version of the system was developed to illustrate common security weaknesses found in web applications, including improper authentication, weak session handling, insecure database queries, and input validation flaws. These vulnerabilities were intentionally introduced to study their impact and to understand how attackers could potentially exploit them. In the subsequent development phases, two hardened versions Non-Vulnerable Version I and Non-Vulnerable Version II were implemented. These versions incorporated appropriate security mechanisms such as secure authentication practices, parameterized queries, encryption of sensitive data, and improved access control policies. The comparative evaluation of the vulnerable and secured versions enabled a deeper understanding of secure coding standards and defensive programming techniques.

The project successfully met its core objectives by demonstrating the entire security cycle from vulnerability identification to mitigation and verification. It also emphasized the importance of incorporating cybersecurity principles during the early stages of software development rather than treating security as an afterthought. Although the team encountered technical challenges during the Docker-based containerization phase, particularly related to configuration conflicts and dependency  and unable to complete it, these difficulties provided valuable insights into secure deployment environments and system isolation techniques.

Overall, the project achieved both its functional and educational goals by integrating theoretical cybersecurity concepts with practical implementation. It strengthened the understanding of web application vulnerabilities, secure software engineering practices, and system hardening methods. The outcomes of this work contribute meaningfully to the learning objectives of applied cybersecurity and underscore the critical role of security awareness in modern software development.