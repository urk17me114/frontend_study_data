%-----------------------------------------------------------------------
\subsection{Django School Management System Implementation}
\label{sec:impl:django-app}
%-----------------------------------------------------------------------

The implementation of the Django School Management System focuses on delivering a modular web application for managing students, teachers, classes, timetables, and announcements. 
The system emphasizes role-based access control, ensuring that Administrators, Teachers, and Students interact with the application according to their permissions. 
The core of the implementation is based on Django's MVC architecture, augmented with third-party libraries for CAPTCHA validation, data parsing, and UI enhancements.

The project is organized into multiple Django apps, each responsible for a specific module:
\begin{itemize}
    \item \texttt{login}: handles user authentication and session management.
    \item \texttt{students}: manages student registration, profiles, and applications.
    \item \texttt{teachers}: handles teacher registration, profiles, and timetable assignments.
    \item \texttt{timetable}: generates and displays class schedules.
    \item \texttt{announcements}: manages posting, viewing, and voting on announcements.
\end{itemize}

\subsubsection{Setup and Environment Requirements}
The project requires Python 3.10+, Django 4.2+, and the following key third-party libraries:

\begin{itemize}
    \item \texttt{django-simple-captcha} for CAPTCHA verification on registration forms.
    \item \texttt{pycountry} to populate nationality choices dynamically.
    \item \texttt{Pillow} for handling profile photo uploads.
    \item \texttt{django-crispy-forms} for form rendering and UI improvements.
\end{itemize}

Database configuration uses SQLite by default, but PostgreSQL is supported for production. 
All migrations must be applied in order, and a superuser must be created for admin-level access. 
The application runs on the built-in Django development server for testing purposes, or via Gunicorn/NGINX in production.

\subsubsection{Differences from Standard Django Conventions}
While the system follows Django's standard MVC patterns, several deviations were introduced to meet the learning objectives:

\begin{itemize}
    \item Custom session decorators were implemented (\texttt{session\_required}) to enforce role-based access.
    \item Certain security vulnerabilities were intentionally included for educational purposes:
        \begin{itemize}
            \item IDOR vulnerability in the Teacher profile view.
            \item CAPTCHA reuse detection demonstration with a limited time window.
            \item Duplicate voting on announcements.
        \end{itemize}
    \item Asynchronous interactions, such as upvoting announcements, are handled via AJAX with JSON payloads instead of traditional page reloads.
    \item Some forms include read-only fields for sensitive data, visually styled to indicate non-editable state.
\end{itemize}

\subsubsection{Supported Functionality and Scope}
The system implements the following key features:

\begin{itemize}
    \item \textbf{Admin:} register new users, approve/reject student and teacher applications, generate class timetables, post announcements, and manage overall system configurations.
    \item \textbf{Teacher:} view assigned timetable, post announcements, upload teaching documents, and update personal profile (with optional photo/document upload).
    \item \textbf{Student:} submit applications, view timetables, access teacher announcements, vote on announcements, and manage personal profile.
    \item \textbf{Security and Validation:} session-based role verification, CAPTCHA validation, input validation on forms, and file type/size checks.
    \item \textbf{Error Handling and Logging:} exceptions during profile updates, file uploads, and downloads are logged to the server console for monitoring.
\end{itemize}

Features partially implemented or left for future work include:
\begin{itemize}
    \item Automated email notifications for student approval or announcement updates.
    \item Multi-class timetable optimization algorithms.
    \item Comprehensive audit logging for all user actions.
\end{itemize}

\subsubsection{Running the System and Known Issues}
To run the system locally:

\begin{enumerate}
    \item Clone the repository and navigate to the project folder.
    \item Install dependencies via \texttt{pip install -r requirements.txt}.
    \item Apply migrations: \texttt{python manage.py migrate}.
    \item Create a superuser for administrative access: \texttt{\seqsplit{python manage.py createsuperuser}}.
    \item Start the development server: \texttt{python manage.py runserver}.
\end{enumerate}

Known limitations and issues include:

\begin{itemize}
    \item CAPTCHA reuse is detected only within a limited time window.
    \item File uploads for photos and documents are limited by default media folder settings; larger files may fail.
    \item Duplicate voting on announcements is possible due to intentional design.
    \item Certain views may raise exceptions if session variables are missing or malformed; these are caught and logged, but some edge cases may result in user-facing errors.
    \item No automated email or notification system has been implemented; notifications rely on manual review.
\end{itemize}
