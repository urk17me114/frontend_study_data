\section{Conclusion and Future Work} 
\label{sec:conclusion-and-future-work} 



While the \textbf{Bobby School of Cyber Security} successfully demonstrates core vulnerabilities and their secure mitigations, there are several avenues for further development and refinement:

\begin{enumerate}
    \item \textbf{Expanded Vulnerability Coverage} \\
    Additional categories of vulnerabilities (e.g., Cross-Site Scripting, Cross-Site Request Forgery, insecure file handling, broken access control) could be implemented to provide a more comprehensive educational experience.

    \item \textbf{Gamification and Challenges} \\
    A structured challenge system could be added, where learners are tasked with finding and exploiting specific vulnerabilities, similar to Capture-The-Flag (CTF) platforms. Points, badges, or leaderboards could further encourage engagement.

    \item \textbf{Integrated Learning Resources} \\
    Explanatory hints, guided tutorials, or links to OWASP documentation could be embedded within the platform, helping learners not only exploit but also understand vulnerabilities.

    \item \textbf{Secure Development Best Practices} \\
    The secure version of the system could be expanded to demonstrate advanced techniques, such as rate-limiting, multi-factor authentication, and secure logging/auditing mechanisms.

    \item \textbf{Scalability and Deployment} \\
    Deployment instructions could be enhanced to support containerization (e.g., Docker) and cloud deployment, making the system easier to distribute and run in classroom or training environments.
\end{enumerate}

\bigskip
By pursuing these improvements, the \textbf{Bobby School of Cyber Security} could evolve into a more versatile, interactive, and scalable platform for teaching secure software development practices.


