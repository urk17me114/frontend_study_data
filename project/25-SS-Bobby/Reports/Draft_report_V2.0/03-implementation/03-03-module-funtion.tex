%-----------------------------------------------------------------------
\section{School Management System Architecture}
\label{sec:impl:system-architecture}
%-----------------------------------------------------------------------

The School Management System is a Django-based web application designed to manage students, teachers, classes, timetables, and announcements. 
The system implements role-based access control, where each participant (Admin, Teacher, Student) interacts with the system through a web interface. 
All communication occurs over standard HTTP/HTTPS, with JSON used for asynchronous operations such as voting on announcements. 
The system leverages Django sessions to maintain authentication and authorization state.

The architecture is modular, separating concerns into dedicated Django apps and views. 
Each module handles specific functionality, such as student registration, timetable management, and announcements. 
Data is stored in a relational database (e.g., SQLite, PostgreSQL) using Django's ORM. 
Security-critical modules, such as CAPTCHA validation and file upload handling, incorporate measures to prevent automated attacks, although some vulnerabilities are intentionally included for learning purposes.

%-----------------------------------------------------------------------
\subsection{Roles and Participants}
\label{subsec:roles-participants}
%-----------------------------------------------------------------------

\subsubsection{Administrator (Admin)}
The Admin is responsible for managing the overall school system. 
Admins have the highest level of access and can perform the following actions:

\begin{itemize}
    \item Add, edit, or delete subjects, teachers, classrooms, and timeslots.
    \item Generate the class timetable based on teacher availability, subjects, and class requirements.
    \item View all timetable entries with filters for class, subject, teacher, and day.
    \item Manage system-level configurations and monitor data integrity.
\end{itemize}

The Admin interacts with the system via the web interface and is authenticated using the \texttt{Admin\_login} session decorator. 
All requests are verified against the session to ensure authorized access.

\subsubsection{Teacher}
Teachers are responsible for delivering lessons and interacting with student-related data. 
Their interactions include:

\begin{itemize}
    \item Viewing their assigned timetable entries for specific days.
    \item Accessing subjects and class information relevant to their teaching assignments.
    \item Posting announcements that students can view and vote on.
    \item Uploading documents or resources, with file type and size validations.
\end{itemize}

Teachers are authenticated using the \texttt{Teacher\_login} session decorator. 
They only see data related to their assignments, ensuring separation of concerns and data privacy.

\subsubsection{Student}
Students are the end users who consume educational content and interact with the system. 
Student actions include:

\begin{itemize}
    \item Registering via a web form with CAPTCHA validation to prevent automated submissions.
    \item Viewing their class timetable and filtering entries by day.
    \item Viewing teacher announcements and voting via upvote/downvote mechanisms.
    \item Tracking personal information, class level, and timetable assignments.
\end{itemize}

Students are authenticated using the \texttt{Student\_login} session decorator. 
Access is restricted to data relevant to their class and assignments.

%-----------------------------------------------------------------------
\subsection{Workflow and Data Flow}
\label{subsec:workflow-dataflow}
%-----------------------------------------------------------------------

The system operates in the following structured workflows:

\begin{itemize}
    \item \textbf{Registration:} Students and Teachers submit registration forms, validated with CAPTCHA and optional file uploads. The system stores validated data in the corresponding \texttt{StudentReg} or \texttt{TeacherReg} tables.
    \item \textbf{Authentication:} Upon login, the system sets session variables to identify the user role (Admin, Teacher, Student) and restrict access to views accordingly.
    \item \textbf{Timetable Generation:} Admin adds subjects, teachers, rooms, and timeslots, then generates the timetable using an algorithm that assigns teachers and rooms to periods. Conflicts are detected and reported.
    \item \textbf{Timetable Viewing:} Students and Teachers fetch timetable entries filtered by their class or assigned schedule. Queries are optimized using Django ORM with related objects.
    \item \textbf{Announcements:} Teachers post announcements. Students view announcements and cast votes, which are tracked in the \texttt{AnnouncementVote} table. JSON is used for asynchronous vote updates.
    \item \textbf{Data Integrity and Security:} The system enforces session-based access control. Modules implement additional validation (CAPTCHA, file type checks). Certain vulnerabilities are intentionally included for educational purposes (CAPTCHA replay, duplicate voting).
\end{itemize}

%-----------------------------------------------------------------------
\subsection{System Communication and Integration}
\label{subsec:system-communication}
%-----------------------------------------------------------------------

All modules communicate internally through Django's ORM and request-response cycle. 
Asynchronous operations, such as voting on announcements, use AJAX with JSON payloads. 
Session management ensures role-specific access control.

\begin{itemize}
    \item Admin, Teacher, and Student communicate via HTTP requests to Django views.
    \item JSON is used for client-server interactions in interactive features (e.g., voting, timetable updates).
    \item ORM queries ensure referential integrity across models like \texttt{Student}, \texttt{Teacher}, \texttt{TimetableEntry}, and \texttt{AnnouncementVote}.
    \item Security-sensitive operations, including registration and voting, are logged and validated against session credentials.
\end{itemize}

%-----------------------------------------------------------------------
\subsection{Security Considerations}
\label{subsec:system-security}
%-----------------------------------------------------------------------

The system-level security considerations include:

\begin{itemize}
    \item Role-based access control using session decorators (\texttt{Admin\_login}, \texttt{Teacher\_login}, \texttt{Student\_login}).
    \item Input validation for forms and file uploads.
    \item CAPTCHA validation for automated bot prevention (with intentional replay vulnerability for study purposes).
    \item Vote integrity concerns, as duplicate votes are currently possible (educational vulnerability).
    \item Optimized ORM queries to prevent unintentional data leaks.
\end{itemize}

