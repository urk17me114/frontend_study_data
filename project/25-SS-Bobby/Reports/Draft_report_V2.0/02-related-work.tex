%-----------------------------------------------------------------------
\chapter{Related Work}
\label{ch:related-work}
%-----------------------------------------------------------------------

The foundation of this project draws heavily from the \textbf{OWASP Juice Shop}, 
an intentionally vulnerable web application developed and maintained by the Open Web Application Security Project (OWASP). 
Juice Shop is widely recognized as one of the most comprehensive training grounds for web application security, 
as it incorporates vulnerabilities that map directly to the \textbf{OWASP Top 10} categories as well as numerous other security flaws encountered in real-world systems.  

The OWASP Juice Shop was not only used as a reference, but also as a hands-on learning platform. 
Students were instructed to attempt exploitation of its vulnerabilities by relying exclusively on the integrated \textit{hint system}, 
rather than consulting the provided solutions. 
This approach was intended to encourage independent exploration, critical thinking, 
and the development of problem-solving strategies commonly used by penetration testers and security researchers.  

Through this practical engagement, students encountered a wide range of vulnerabilities including 
\textit{injection attacks}, 
\textit{broken authentication}, 
\textit{insecure direct object references (IDOR)}, 
and \textit{security misconfigurations}. 
These exercises provided both a theoretical grounding in web security concepts 
and a practical appreciation of how attackers exploit insecure code in real-world contexts.  

By incorporating the OWASP Juice Shop into the project’s foundation, 
the work is situated within a well-established security education framework. 
It also ensures that the deliberately insecure web application developed in this project, 
named the \textbf{Bobby School of Cyber Security}, 
reflects realistic vulnerabilities, thereby strengthening its effectiveness as a teaching tool 
for secure software engineering practices.  






%Related literature \cite{DBLP:journals/cacm/RivestSA78}. \lipsum[2]