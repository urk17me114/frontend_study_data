\documentclass{beamer}
\usetheme{Madrid}
\usepackage{graphicx} 
\usepackage{listings}
\usepackage{xcolor}



\title{Vulnerabilities in OWASP Juice Shop}
\author{Prepared by Ganga Sunil}
\date{May 23, 2025}

\begin{document}

\begin{frame}
  \titlepage
\end{frame}

\begin{frame}[fragile]{Log in with the Support Team’s Original User Credentials}

\textbf{Steps:}
\begin{itemize}
  \item Support team email: \texttt{support@juice-sh.op} from SQL Injection user list.
  \item Locate KeePass database file at: \\
  \url{http://localhost:3000/ftp/incident-support.kdbx}
  \item Download and open with KeePass 2.x.
  \item From the hintspassword to the keypass file is weak and start with Support, brute force the remaining to Support2022! to unlock KeePass file.
  \item Find password for support user in the \texttt{prod} entry inside KeePass.
  \item Use these credentials to log in.


\end{itemize}

\end{frame}

\begin{frame}[fragile]{Log in with the Support Team’s Original User Credentials}
\begin{figure}
    \centering
    \includegraphics[width=0.75\linewidth]{keypass.jpg}
    \caption{Passwords in keypass file.}
    \label{fig:enter-label}
\end{figure}
\end{frame}
\begin{frame}[fragile]{Perform a Remote Code Execution to Keep the Application Busy Forever}

\textbf{Step-by-step guide:}

\begin{itemize}
  \item From the hint looking for doccument related to apis leads to Swagger API docs at \url{http://localhost:3000/api-docs} describing the B2B API.
  \item The API allows POSTing orders with \texttt{orderLines} in JSON.
  \item Authenticate with any bearerAuth rextracted during login by burp suite.
  \item Example \texttt{orderLinesData} input contains arbitrary JSON:
  \begin{lstlisting}[basicstyle=\ttfamily\footnotesize]
 {  "orderLinesData": "'); while(true){} //"}
  \end{lstlisting}

  \item Click “Try it out” then “Execute” for a successful \texttt{200} response.
  \item The server will respond after about 2 seconds due to a timeout, preventing a true DoS.
\end{itemize}

\end{frame}

\begin{frame}[fragile]{Perform a Remote Code Execution to Keep the Application Busy Forever}
\begin{figure}
    \centering
    \includegraphics[width=0.40\linewidth]{swagger1.jpg}
    \caption{Infinite loop JSON command in swagger.}
    \label{fig:enter-label}
\end{figure}

\end{frame}


\begin{frame}[fragile]{Perform Remote Code Execution Using Costly Regular Expression}

\begin{itemize}
    \item Follow steps in the above challenge.
    \item In the \textbf{Request Body}, replace the example with a heavy equation:
    \begin{lstlisting}[basicstyle=\ttfamily\small]
    { 
  "orderLinesData": 
  "/((1+)+)5/.test('11111111111111111111111111')"
}
    \end{lstlisting}
    \item This triggers a \textbf{catastrophic backtracking} in regex evaluation, a very costly operation.
    \item Click \textbf{Execute} to send the request.
    \item Server will respond with a \textbf{503 Service Unavailable} and message:

        \textit{Sorry, we are temporarily not available! Please try again later.}

\end{itemize}

\end{frame}

\begin{frame}[fragile]{Perform Remote Code Execution Using Costly Regular Expression}
\begin{figure}
    \centering
    \includegraphics[width=0.40\linewidth]{swagger2.jpg}
    \caption{JSON command in swagger.}
    \label{fig:enter-label}
\end{figure}
\end{frame}

\begin{frame}[fragile]{Inform the shop about a typosquatting trick it has been a victim of at least in v6.2.0-SNAPSHOT.}
\begin{itemize}
    \item After solving \textbf{Download and inspect backup file}, \texttt{package.json.bak} dependencies.
    \item Use online tool https://snyk.io/advisor/check/npm to inspect the libraries for typosqatting.
    \item Which reveals epilouge-js is a typosquatting package mimicking the legitimate \texttt{epilogue} library.
    
\end{itemize}
\end{frame}
\begin{frame}
\begin{figure}
    \centering
    \includegraphics[width=0.5\linewidth]{image.png}
    \caption{Epilogue-Js Library}
    \label{fig:enter-label}
\end{figure}
\end{frame}
\begin{frame}[fragile]{Informing the Shop About Vulnerable Libraries}
\begin{itemize}
    \item Juice Shop depends on JavaScript libraries with known vulnerabilities.
    \item Use online tool https://snyk.io/advisor/check/npm to inspect the
libraries.
    \item Notable vulnerable packages found:
    \begin{itemize}
        \item \textbf{sanitize-html} (version 1.4.2): 
        \begin{itemize}
            \item HTML sanitization not recursive, allowing script injection.

        \end{itemize}
        
    \end{itemize}
    \item Submit feedback via \url{http://localhost:3000/#/contact} including:
    \begin{itemize}
        \item \texttt{sanitize-html} and \texttt{1.4.2}
    \end{itemize}
    \item This informs the shop about its vulnerable dependencies to fix.
\end{itemize}
\end{frame}


\begin{frame}[fragile]{Informing the Shop About Vulnerable Libraries}

\begin{figure}
        \centering
        \includegraphics[width=0.5\linewidth]{sanitize.jpg}
        \caption{Vulnerable sanitize-html Library}
        \label{fig:enter-label}
    \end{figure}
        
\end{frame}



\begin{frame}[fragile]{Permanently Disable the Support Chatbot}
\textbf{Key points:}
\begin{itemize}
    \item The chatbot uses the npm module \texttt{juicy-chat-bot}.
    \item User messages are processed inside a VM context via a \texttt{process} function.
    \item Vulnerable code snippet:  
         this.factory.run(`users.addUser("${token}", "${name}")`)

    This acts like an \texttt{eval} inside the VM.
    
    \item Injecting username: \texttt{admin"); process=null;//}                                      
    \item The executed code becomes:
    \begin{lstlisting}
        users.addUser("token", admin");
        process=null; //
    \end{lstlisting}
    
    \item Setting \texttt{process = null} disables message processing.
    \item Subsequent chatbot interactions will cause errors, effectively disabling the bot.
\end{itemize}



\end{frame}
\begin{frame}[fragile]{Permanently Disable the Support Chatbot}

\begin{figure}
    \centering
    \includegraphics[width=0.5\linewidth]{Screenshot 2025-05-22 153348.jpg}
    \caption{Disabled chatbot}
    \label{fig:enter-label}
\end{figure}
\end{frame}

\end{document}


