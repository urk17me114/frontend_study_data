\documentclass{beamer}
\usetheme{Madrid}
\usepackage[utf8]{inputenc}
\usepackage{listings}
\usepackage{hyperref}
\usepackage{graphicx}

\title{Teacher Profile and Pinboard Security Vulnerabilities}
\author{Thoomu Sai Bhargav}
\date{\today}

\begin{document}

\begin{frame}
    \titlepage
\end{frame}

%=============================
\section{Cross-Site Scripting (XSS)}
%=============================

\begin{frame}[fragile]{Cross-Site Scripting (XSS) in Pinboard Comments}
Stored Cross-Site Scripting (XSS) occurs when user-submitted comments are rendered without proper sanitization, allowing attackers to inject malicious scripts.
\begin{itemize}
    \item Only admins can post announcements; students and teachers can comment.
    \item If HTML input does not escape, attackers can embed JavaScript in comments.
    \item The script executes in any viewer’s browser, compromising their session.
    \item Since the payload is stored in the database, it affects all viewers persistently.
\end{itemize}
\end{frame}

\begin{frame}[fragile]{Comment Functionality}
\begin{itemize}
    \item Students and teachers can post comments on admin announcements.
    \item Only authenticated users are allowed to comment (enforced via session decorators).
    \item Comments are submitted via POST requests:
    \begin{itemize}
        \item Includes announcement ID and comment text.
    \end{itemize}
    \item Backend stores the comment and renders it with the announcement.
    \item No HTML filtering or escaping is applied.
    \item Any embedded \texttt{<script>} tags or malicious HTML will execute in the browser.
\end{itemize}
\end{frame}

\begin{frame}[fragile]{Vulnerability Walkthrough}
\textbf{Step 1: Test Payload Injection}
\begin{itemize}
    \item Log in as a student.
    \item Submit a comment with:
\begin{lstlisting}
<script>alert('XSS Test')</script>
\end{lstlisting}
    \item View the announcement.
    \item If an alert box pops up, XSS is confirmed.
\end{itemize}
\textbf{Step 2: Escalate the Attack}
\begin{itemize}
    \item Inject a fake login form using HTML and JavaScript.
    \item When other users view the infected comment:
    \begin{itemize}
        \item The form appears.
        \item Credentials entered are captured by the attacker.
    \end{itemize}
    \item Browsers trust and execute the content in the logged-in user’s context.
\end{itemize}
\end{frame}

\begin{frame}[fragile]{Exploit Setup – Attacker Server}
\begin{itemize}
    \item Attacker sets up a local server to capture credentials:
    \begin{itemize}
        \item Listens on port 8080
        \item Logs POST data to \texttt{fake\_login.log}
    \end{itemize}
\end{itemize}
\end{frame}

\begin{frame}[fragile]{Exploit Payload – Fake Login Form}
\begin{itemize}
    \item When users view the comment:
    \begin{itemize}
        \item The fake form appears.
        \item Submitted credentials are silently logged to \texttt{fake\_login.log}.
    \end{itemize}
\end{itemize}
\end{frame}

\begin{frame}[fragile]{XSS – Consequences}
\begin{itemize}
    \item Credential theft: users unknowingly submit sensitive data.
    \item Session hijacking: attacker gains access to user accounts.
    \item UI manipulation: attacker defaces interface or misleads users.
    \item Persistent threat: stored XSS affects all future viewers.
    \item Reputation damage: users lose trust in the platform.
    \item Legal risk: violates data protection laws.
    \item Social engineering: attacker impersonates staff or redirects users.
\end{itemize}
\end{frame}

\begin{frame}[fragile]{Possible Mitigations}
\begin{itemize}
    \item Escape HTML using Django’s autoescape to prevent script injection.
    \item Sanitize input with libraries like Bleach or html-sanitizer before saving.
    \item Apply a Content Security Policy (CSP) to restrict script execution and external resources.
    \item Treat all comment input as plain text to avoid rendering raw HTML.
    \item Whitelist safe characters and markup only.
    \item Log and monitor comment activity for suspicious patterns.
    \item Educate developers on secure coding practices.
    \item Regularly audit user-generated content for vulnerabilities.
\end{itemize}
\end{frame}

%=============================
\section{IDOR – Teacher Profile}
%=============================

\begin{frame}[fragile]{Insecure Direct Object Reference (IDOR) in Teacher Profile}
\begin{itemize}
    \item IDOR occurs when the app exposes internal objects without verifying authorization.
    \item The Teacher Profile module accepts a username query parameter and returns the profile page without checking if the requested username matches the logged-in user.
    \item Allows authenticated teachers to access or modify other teacher’s private data.
    \item Leads to data exposure, impersonation, and unauthorized changes.
\end{itemize}
\end{frame}

\begin{frame}[fragile]{View and Manage Teacher Profile Functionality}
\begin{itemize}
    \item Teachers can view and update their own profile information.
    \item Only authenticated teachers can access the profile page.
    \item Editable fields: address, certification, profile photo, documents.
    \item Email and birth date are read-only.
    \item Uploaded files must be under 1 MB.
    \item Changes are saved to the database and reflected on the platform.
    \item No server-side validation is enforced on the username parameter.
\end{itemize}
\end{frame}

\begin{frame}[fragile]{IDOR Vulnerability Walkthrough}
\textbf{Step 1: Access Own Profile}
\begin{itemize}
    \item Log in as a teacher.
    \item Visit: \texttt{http://127.0.0.1:8000/login/teacher/profile/?username=TPX627}
\end{itemize}
\textbf{Step 2: Modify the Query Parameter}
\begin{itemize}
    \item Change the username parameter to another teacher ID, e.g., TPX999.
    \item Reload the page.
    \item If another teacher’s profile is visible, the endpoint is vulnerable.
\end{itemize}
\textbf{Step 3: Modify Another Teacher’s Data}
\begin{itemize}
    \item Edit fields such as address, certification, and upload new documents.
    \item Changes overwrite legitimate teacher’s data.
    \item Attacker has full control over another user’s profile without authorization.
\end{itemize}
\end{frame}

\begin{frame}[fragile]{IDOR Exploit Setup – Attacker Behavior}
\begin{itemize}
    \item Attacker is a logged-in teacher.
    \item Modifies the username parameter in the URL.
    \item Server does not validate ownership of the profile.
    \item Gains access to another teacher’s personal info and can modify it.
    \item Used for impersonation, phishing, or data tampering.
\end{itemize}
\end{frame}

\begin{frame}[fragile]{IDOR – Consequences}
\begin{itemize}
    \item Unauthorized access to private teacher data.
    \item Exposure of personal documents and profile photos.
    \item Unauthorized modification of another teacher’s profile.
    \item Potential for impersonation or social engineering.
    \item Breach of confidentiality and trust.
    \item Violation of data protection regulations.
    \item Risk of further exploitation if sensitive files are altered or replaced.
\end{itemize}
\end{frame}

\begin{frame}[fragile]{IDOR – Possible Mitigations}
\begin{itemize}
    \item Enforce server-side ownership: derive target user from session, not client input.
    \item Apply access-control checks before showing or modifying other users’ data.
    \item Use indirect identifiers like UUIDs or signed tokens instead of usernames.
    \item Remove sensitive data from URLs; use session-based routes like /profile/me.
    \item Harden sessions with secure cookies, short lifetimes, and rotation.
    \item Validate file uploads with strict server-side checks.
    \item Log and monitor access patterns to detect cross-user activity.
\end{itemize}
\end{frame}

%=============================
\section{CSRF – Teacher Profile}
%=============================

\begin{frame}[fragile]{Cross-Site Request Forgery (CSRF) in Teacher Profile}
\begin{itemize}
    \item CSRF tricks logged-in users into performing actions without consent.
    \item Browsers automatically send session cookies; forged requests appear legitimate.
    \item Teacher profile endpoint disables CSRF using \texttt{@csrf\_exempt}.
    \item POST requests update profile fields silently.
    \item Enables unauthorized profile manipulation and sensitive actions.
\end{itemize}
\end{frame}

\begin{frame}[fragile]{CSRF Walkthrough}
\textbf{Step 1: Prepare the Attack}
\begin{itemize}
    \item Save a malicious HTML file (attacker.html) that auto-submits a form to teacher profile endpoint.
    \item Form includes hidden fields to change profile data (firstname, lastname).
\end{itemize}
\textbf{Step 2: Host the Attack Page}
\begin{itemize}
    \item Run local server: \texttt{python -m http.server 8001}.
    \item Serve \texttt{attacker.html} from the server.
\end{itemize}
\textbf{Step 3: Trigger the Attack}
\begin{itemize}
    \item Visit \texttt{http://127.0.0.1:8001/attacker.html} while logged in.
    \item Page auto-submits POST request.
    \item If CSRF disabled, profile silently updated.
\end{itemize}
\end{frame}

\begin{frame}[fragile]{CSRF Exploit Setup – Attacker Behavior}
\begin{itemize}
    \item Craft malicious HTML page with hidden form targeting teacher profile update endpoint.
    \item Victim visits the page; browser auto-submits form.
    \item Server accepts request without verifying origin or CSRF token.
    \item Victim’s profile modified without knowledge.
\end{itemize}
\end{frame}

\begin{frame}[fragile]{CSRF – Consequences}
\begin{itemize}
    \item Unauthorized modification of teacher profile data.
    \item Loss of control over personal information.
    \item Potential for impersonation or misinformation.
    \item Breach of trust and platform integrity.
    \item Violation of data protection regulations.
    \item Risk of further exploitation if extended to other endpoints.
\end{itemize}
\end{frame}

\begin{frame}[fragile]{CSRF – Possible Mitigations}
\begin{itemize}
    \item Restore CSRF protection: remove \texttt{@csrf\_exempt}, enable CsrfViewMiddleware.
    \item Include \texttt{\{\% csrf\_token \%\}} in all HTML forms.
    \item Support AJAX by sending CSRF tokens via X-CSRFToken header.
    \item Configure cookies with SameSite=Lax/Strict and Secure flag.
    \item Validate request origin using Origin or Referer headers.
    \item Accept only POST for state-changing actions.
    \item Require re-authentication for sensitive updates.
\end{itemize}
\end{frame}

%=============================
\section{Unrestricted File Upload – Teacher Profile}
%=============================

\begin{frame}[fragile]{Unrestricted File Upload in Teacher Profile}
\begin{itemize}
    \item Occurs when uploads are not validated (type, size, content).
    \item Allows oversized files, unexpected types, or executable files (.php).
    \item Consequences: DoS (disk exhaustion), data corruption, info disclosure, remote code execution.
\end{itemize}
\end{frame}

\begin{frame}[fragile]{File Upload Functionality}
\begin{itemize}
    \item Teachers can view and update profile information.
    \item Only authenticated teachers can access the profile page.
    \item Editable: address, certification, profile photo, documents.
    \item Email and birth date are read-only.
    \item Uploaded files intended <1 MB, but no server-side validation.
    \item Files stored on server, may be accessible via predictable URLs.
\end{itemize}
\end{frame}

\begin{frame}[fragile]{File Upload Vulnerability Walkthrough}
\textbf{Step 1: Upload Oversized File}
\begin{itemize}
    \item Log in, select PDF >1 MB, intercept via Burp Suite, forward request.
\end{itemize}
\textbf{Step 2: Upload Unauthorized File Type}
\begin{itemize}
    \item Select PNG or other type, modify filename/Content-Type, forward request.
\end{itemize}
\textbf{Step 3: Upload Executable File}
\begin{itemize}
    \item Prepare PHP file (test.php), intercept, replace filename/content, forward request.
    \item If executed, remote code execution possible.
\end{itemize}
\end{frame}

\begin{frame}[fragile]{File Upload Exploit Setup – Attacker Behavior}
\begin{itemize}
    \item Logged-in teacher intercepts and modifies upload requests via Burp Suite.
    \item Bypass frontend restrictions by changing headers/filenames.
    \item Upload oversized or unauthorized files.
    \item If server stores/serves files without validation, attacker can access/execute them remotely.
\end{itemize}
\end{frame}

\begin{frame}[fragile]{File Upload – Consequences}
\begin{itemize}
    \item Disk exhaustion due to large/repeated uploads.
    \item Exposure of unauthorized/sensitive content.
    \item Remote code execution.
    \item Data corruption or overwrite.
    \item Breach of confidentiality and system integrity.
    \item Violation of data protection/security policies.
\end{itemize}
\end{frame}

\begin{frame}[fragile]{File Upload – Possible Mitigations}
\begin{itemize}
    \item Enforce server-side size limits (e.g., 1 MB).
    \item Accept only whitelisted MIME types, verify file signatures.
    \item Sanitize filenames, use randomized names (UUIDs).
    \item Store uploads outside web root, disable script execution.
    \item Scan files before permanent storage.
    \item Apply least-privilege access, restrict filesystem permissions.
    \item Serve files via signed URLs or authenticated endpoints.
    \item Log upload metadata and trigger alerts on suspicious activity.
\end{itemize}
\end{frame}

\end{document}
