\documentclass{beamer}
\usetheme{Madrid}
\usepackage{graphicx}
\usepackage[utf8]{inputenc}

\title{Bobby School of Cyber Security}

\begin{document}

% --- Title Slide ---
\begin{frame}
    \titlepage
    \begin{figure}
        \centering
        \includegraphics[width=\linewidth]{bobby_welcome_bg.jpg}
    \end{figure}
\end{frame}


%-------------------------------------------------------
\begin{frame}{SQL Injection}
SQL Injection occurs when user input is interpreted as SQL code by the database rather than plain data. 

\vspace{0.3cm}

\begin{itemize}
    \item Concatenating raw user input directly into SQL queries.
    \item Missing parameterization or prepared statements.
    \item Insufficient server-side input validation and sanitization.
\end{itemize}
\end{frame}

%-------------------------------------------------------

\begin{frame}{SQL Injection — Consequences}
\textbf{Impact:}
\begin{itemize}
    \item Unauthorized retrieval of sensitive data (e.g., student records, passwords).
    \item Modification or deletion of database records.
    \item Bypass of authentication mechanisms.
    \item Complete compromise of application and database integrity.
\end{itemize}


\end{frame}

%-------------------------------------------------------

%-------------------------------------------------------
\begin{frame}{Search Workflow}

\begin{itemize}
    \item Displays a search form to the user.
    \item Accepts a name query via POST request.
    \item Executes a raw SQL query to find matches.
    \item Retrieves results as dictionaries and renders the results.
\end{itemize}

\end{frame}
%-------------------------------------------------------




% Slide: Reproduction steps (quick)
\begin{frame}{Vulnerebility}
    \begin{enumerate}
        \item Open Student / Teacher dashboard search page.
        \item Confirm the existence if the sql injection by using ' OR 1=1; -- .
        \item Submit a UNION SELECT payload  into the search box.
        \item By trial and error on selected columns until the UNION succeeds.
        \item Use payloads on information\_schema.tables and information\_schema.columns to get table and column names to extract sensitive data.
        \item Inspect search result table for injected values.
    \end{enumerate}
\end{frame}
%-------------------------------------------------------
\begin{frame}{Vulnerebility}
\begin{table}[h!]
\centering
\renewcommand{\arraystretch}{1.3}
\begin{tabular}{|p{6.5cm}|p{4cm}|}
\hline
\textbf{Payloads} & \textbf{Use / Purpose} \\ \hline
\texttt{a' UNION SELECT table\_name, 'x','x','x','x','x','x','x' FROM information\_schema.tables; --} & Retrieve the names of all tables in the database. \\ \hline
\texttt{a' UNION SELECT column\_name, 'x','x','x','x','x','x','x' FROM information\_schema.columns WHERE table\_name = 'login\_login'; --} & Retrieve column names of the table \texttt{login\_login}. \\ \hline
\texttt{a' UNION SELECT username, password, 'x','x','x','x','x','x' FROM login\_login; --} & Extract actual data (usernames and passwords) from the \texttt{login\_login} table. \\ \hline
\end{tabular}
\end{table}
\end{frame}
%-------------------------------------------------------
%-------------------------------------------------------
\begin{frame}{Impact}
\small
\begin{itemize}
    \item \textbf{Confidentiality breach:} Personal data, grades, and credentials can be exfiltrated.
    \item \textbf{Integrity loss:} Attackers may alter grades, attendance, or records and corrupting academic outcomes.
    \item \textbf{Authentication compromise:} Stolen credentials enable account takeover and wider access.
    \item \textbf{Availability / disruption:} Malicious queries or cleanup activities can cause downtime or degraded service.
    \item \textbf{Academic integrity risk:} Exams, assignments or results can be tampered with, undermining fairness.
    \item \textbf{Financial \& operational cost:} Incident response, recovery, and possible fines create significant expense.
   
    
\end{itemize}
\end{frame}
%-------------------------------------------------------

%-------------------------------------------------------

\begin{frame}{Possible Mitigations}


\begin{itemize}
    \item \textbf{Use parameterized queries / prepared statements:} Never build SQL by string concatenation, always bind parameters.
    \item \textbf{Prefer an ORM or safe query builder:} ORMs (or query builders) automatically parameterize and reduce direct SQL usage.
    \item \textbf{Validate and whitelist inputs:} Enforce allowed formats or lengths and use whitelists where possible.
    \item \textbf{Limit database privileges:} Give the application account the minimum permissions required.
    \item \textbf{Hide DB errors from clients:} Return generic error messages to users and log full error details server-side for investigation.
 


\end{itemize}
\end{frame}








%-------------------------------------------------------
\begin{frame}{IDOR (Insecure Direct Object Reference)}
An Insecure Direct Object Reference (IDOR) occurs when an application exposes 
internal identifiers without proper server-side authorization checks.

\begin{itemize}
    \item Server trusts client-supplied identifiers to decide which data to display.
    \item Missing or improper authorization checks before accessing resources.
    \item Predictable or sequential identifiers make guessing easy.
    \item Lack of separation between user input and data ownership verification.
\end{itemize}
\end{frame}


%-------------------------------------------------------

\begin{frame}{IDOR Consequences}
\textbf{Impact:}
\begin{itemize}
    \item Unauthorized access to sensitive information.
    \item Modification or deletion of other users resources.
    \item Privacy violations and data integrity loss.
    \item Potential legal and reputational damage to the organization.
\end{itemize}

\end{frame}

%-------------------------------------------------------
%-------------------------------------------------------
\begin{frame}{Marks Management Module}

\textbf{Add Marks - Teacher}
\begin{itemize}
    \item Teachers select class, subject, exam type, exam date, and total marks before entering student results.
    \item Allows teachers to input marks for all students in the chosen class and subject. 
    \item Prevents duplicate entries and validates inputs.
\end{itemize}

\vspace{0.3cm}
\textbf{Student Workflows}
\begin{itemize}
    \item Displays marks for the logged-in student with filtering by subject or class section and sorting by marks.
    \item Updates the \texttt{student\_username} session variable to synchronize local storage with the server session.
\end{itemize}
\end{frame}
%-------------------------------------------------------



\begin{frame}[fragile]{Vulnerebility}
    \begin{itemize}
        \item Log in as a valid student and open your marks page.
        
        \item Open browser DevTools \(\rightarrow\) Application (or Storage) tab.
        \item Inspect \texttt{sessionStorage} for keys  \texttt{student\_username}.
            \end{itemize}
            \begin{figure}
            \centering
            \includegraphics[width=.9\linewidth]{image4.png}
            \caption{View Marks}
            \label{fig:placeholder}
        \end{figure}

\end{frame}
\begin{frame}[fragile]{Vulnerebility}
    \begin{itemize}

        \item Edit the value to another valid student's username.
        \item Refresh the marks page to displays the other student's marks.
        \begin{figure}
            \centering
            \includegraphics[width=0.9\linewidth]{image.png}
            \caption{Other syidents mark}
            \label{fig:placeholder}
        \end{figure}
    \end{itemize}
    \vspace{0.5em}
    \textit{Cause: The session storage key is reffered by the database to dispalys the marks.}
\end{frame}

\begin{frame}{Possible Mitigation}
    \begin{itemize}
        \item \textbf{Server-side authorization:} Always verify on the server that the authenticated user is allowed to access the requested resource.
        \item \textbf{Do not trust client-side identifiers:} Treat client-supplied IDs as untrusted input.
        \item \textbf{Use opaque references or verify ownership:} Map public identifiers to internal records and check ownership/roles per request.
        \item \textbf{Least privilege / RBAC:} Enforce role-based checks for each endpoint.

        \item \textbf{Logging \& alerts:} Log suspicious attempts to access others' records and alert on anomalies.
    \end{itemize}
\end{frame}

%----------
%-------------------------------------------------------




%-------------------------------------------------------
\begin{frame}{Sensitive Data Exposure by Insecure RBAC}
Sensitive Data Exposure due to Insecure Role-Based Access Control occurs when 
an application decides user privileges using client-controlled inputs, rather than secure server-side role checks.  
\vspace{0.3cm}

\begin{itemize}
    \item Application trusts client-side input to determine access level.
    \item Missing server-side enforcement of user roles or permissions.
    \item Inconsistent or incomplete role verification in backend logic.
    \item Lack of session-based privilege management or centralized access control mechanism.
\end{itemize}
\end{frame}
%-------------------------------------------------------

\begin{frame}{Insecure RBAC — Examples}
\textbf{Common Examples:}
\begin{itemize}
    \item Using query parameters or cookies to determine access.
    \item Relying on hidden form fields or JavaScript variables to decide if a user is an admin.
    \item Exposing sensitive data or settings pages through weak URL-based authorization.

\end{itemize}


\end{frame}

%-------------------------------------------------------

\begin{frame}{Insecure RBAC — Consequences}
\textbf{Impact:}
\begin{itemize}
    \item Unauthorized users can access confidential data or admin dashboards.
    \item Exposure of audit logs, configuration files, or user data.
    \item Privilege escalation and complete system compromise.
    \item Violations of privacy and compliance standards.
\end{itemize}


\end{frame}
%-------------------------------------------------------



%-------------------------------------------------------
\begin{frame}{View Logs}

\begin{itemize}
    \item Allows administrators to access login activity logs.
    \item Validates that the requesting user has admin privileges before serving the logs.
    \item Validation done by checking a flag that keep the role of the user.
    \item Unauthorized access attempts are blocked and return an error.
\end{itemize}
\end{frame}
%-------------------------------------------------------

\begin{frame}[fragile]{Vulnerebility}
    \begin{itemize}
        \item Attempting to open \verb|http://127.0.0.1:8000/login/view-logs| will give error page due to lack of access.
        \begin{figure}
            \centering
            \includegraphics[width=0.8\linewidth]{image5.png}
            \caption{Error page}
            \label{fig:placeholder}
        \end{figure}
    \end{itemize}
    \end{frame}





\begin{frame}[fragile]{Vulnerebility}

    \begin{itemize}

        \item If denied, append the parameter: \\
              \verb|http://127.0.0.1:8000/login/view-logs?admin=1|
              \begin{figure}
                  \centering
                  \includegraphics[width=0.8\linewidth]{image6.png}
                  \caption{Log page}
                  \label{fig:placeholder}
              \end{figure}
    \end{itemize}
    \vspace{0.5em}
    \textit{Cause: The access is granted by the endpoint relying on the session flag for privilege decision}

\end{frame}

\begin{frame}{Mitigation — Secure Role & Access Checks}
    \begin{itemize}
        \item \textbf{Server-authoritative roles:} Do not use request parameters to determine privileges. Check user role stored on the server-side (session / token).
        \item \textbf{Strict RBAC enforcement:} Centralize access control checks (middleware or decorators) for sensitive endpoints.
        \item \textbf{Avoid security-by-obscurity:} Do not rely on hidden fields, client flags, or unprotected query parameters for access control.
        \item \textbf{Use POST with CSRF for state-changing ops:} Avoid using GET parameters for actions that alter behavior or leak sensitive data.
        \item \textbf{Sanitize & log:} Log attempts to access admin endpoints without proper rights and alert on suspicious patterns.

    \end{itemize}
\end{frame}


%-------------------------------------------------------
\begin{frame}{Security Misconfiguration}
Security misconfiguration occurs when a system, server, or application is set up insecurely.
    \vspace{0.5em}

\begin{itemize}
    \item Default configurations not changed after installation.
    \item Exposed debug or error pages revealing sensitive system information.
    \item Missing authorization checks on certain URLs or API functions.
    \item Unsecured endpoints.
    \item Enabling unnecessary services, modules, or verbose logging.
\end{itemize}
\end{frame}
%-------------------------------------------------------

\begin{frame}{Security Misconfiguration — Examples}
\textbf{Common Examples:}
\begin{itemize}
    \item Leaving \texttt{default usernames/passwords} unchanged.
    \item Keeping \texttt{debug mode} or developer tools enabled in production.
    \item Failing to restrict access to admin panels or configuration files.
    \item Running software with excessive permissions or unpatched components.
\end{itemize}

\end{frame}


\begin{frame}{Security Misconfiguration — Consequences}
\textbf{Potential Consequences:}
\begin{itemize}
    \item Attackers bypass authentication or authorization mechanisms.
    \item Exposure of sensitive data such as database credentials, environment files.
    \item Unauthorized code execution or privilege escalation.
    \item Complete compromise of the application or server.
\end{itemize}

\end{frame}


%-------------------------------------------------------
\begin{frame}{Forgot Password Process}
\textbf{Step 1}
\begin{itemize}
    \item Clears any previous session data.
    \item Accepts and verifies if the username.
\end{itemize}

\vspace{0.2cm}
\textbf{Step 2}
\begin{itemize}
    \item Prompts the user for their security answer using username session.
    \item If correct, sets a session flag and allows password reset access.
    \item If incorrect, redirects to the login and flush the session.
    \item Reset the target user’s password without answering any verification
question.
\end{itemize}

\vspace{0.2cm}
\textbf{Step 3}
\begin{itemize}
    \item Confirms valid username.
    \item Accepts and validates new password inputs.
    \item On success, clears session flags and redirects to the Login page.
\end{itemize}
\end{frame}
%-------------------------------------------------------


% --- Slide: Solution / reproduction (step-by-step) ---
\begin{frame}[fragile]{Vulnerebility}
    \begin{itemize}
        \item Open the Forgot Password page and enter the target username.
        \begin{figure}
            \centering
            \includegraphics[width=0.5\linewidth]{image1.png}
            \caption{Forgot Password: Step 1}
            \label{fig:placeholder}
        \end{figure}
    \end{itemize}
\end{frame}
\begin{frame}[fragile]{Vulnerebility}
\begin{itemize}
        \item For users without a security question, the app redirects back to login.
\begin{figure}
    \centering
    \includegraphics[width=0.43\linewidth]{image2.png}
    \caption{Redirect to Login page}
    \label{fig:placeholder}
\end{figure}
    \end{itemize}

\end{frame}
\begin{frame}[fragile]{Vulnerebility}
\begin{itemize}

        \item Manually navigate to the reset step URL: \\ \verb|/forgot-password/step3|
        \item If the endpoint lacks server-side verification, you can set a new password for that user.
\end{itemize}
\begin{figure}
    \centering
    \includegraphics[width=0.3\linewidth]{image3.png}
    \caption{Reset Password page}
    \label{fig:placeholder}
\end{figure}

    
    \textit{Cause: When security question is not set username session is not cleared during redirect.}

\end{frame}

% --- Slide: Evidence & impact ---
\begin{frame}{Effects and Impact}
    \begin{itemize}
        \item \textbf{Effect:} Able to reach reset page and change password without completing proper security check.
        \vspace{0.2cm}

        \item \textbf{Impact:}
            \begin{itemize}
                \item Account takeover of an another users.
                \item Potential data exposure and regulatory risk.
                \item Enables further attacks such as privilege escalation and fraud.
            \end{itemize}
    \end{itemize}
\end{frame}

% --- Slide: Mitigations (how to fix) ---
\begin{frame}{Possible Mitigations}
    \begin{itemize}
        \item \textbf{Session validation:} Each forgot-password step must verify a server-side authentication that proves previous steps completed.
        \item \textbf{Use time-limited tokens:} Password reset flows generate unique, expiring session tokens for each request to prevent direct URL manipulation.
        \item \textbf{Do not rely on client-side state:} Never trust redirects, query parameters or client storage to indicate progress.
        \item \textbf{Validate presence of security data:} If a user lacks a security question, require alternative verification.
        \item \textbf{Rate-limit and log:} Limit reset attempts and log suspicious attempts.
       
    \end{itemize}
\end{frame}



%-------------------------------------------------------
\begin{frame}{XXE - XML External Entity}
XXE occurs when an XML parser processes external entities defined in the XML document, allowing attackers to access local files, network resources, or trigger denial-of-service via entity expansion.  
\vspace{0.3cm}

\begin{itemize}
    \item XML parsers allow DTD processing by default.
    \item External entity resolution is enabled.
    \item Lack of input validation on XML content from untrusted sources.
    \item Use of unsafe XML libraries without hardening or security configuration.
\end{itemize}
\end{frame}


%-------------------------------------------------------

\begin{frame}{XXE — Consequences}
\textbf{Impact:}
\begin{itemize}
    \item Exposure of sensitive files (e.g., passwords, configuration, student data).
    \item Remote requests triggered from the server (SSRF).
    \item Denial-of-service via entity expansion (e.g., Billion Laughs attack).
    \item Compromise of data integrity and application availability.
\end{itemize}

\end{frame}

%-------------------------------------------------------



%-------------------------------------------------------
\begin{frame}{Assignment Management module}
\small

\textbf{Teacher Workflows}
\begin{itemize}
    \item Post questions limited to the teacher's assigned subjects and class sections.
    \item View all questions authored by the teacher.
    \item Filter submissions by subject, class section and status; order by submission time.
    \item Assign marks and set submission status to graded.
\end{itemize}

\vspace{0.25cm}
\textbf{Student Workflows}
\begin{itemize}
    \item Show questions available for the student's class and subjects.
    \item Upload any type of file with text as an answer.
    \item View past submissions with status and marks.
\end{itemize}

\end{frame}
%-------------------------------------------------------



\begin{frame}{Vulnerebility}
\begin{enumerate}
    \item Login to the dashboard using a student account.
    \item Navigate to My Questions for the assignment section.
    \item Select the respective assignment question.
    \item Attempt to upload an XML file designed to test entity expansion.
    \item On downloading the file from My submission activate the billion laugh
\end{enumerate}
\begin{figure}
    \centering
    \includegraphics[width=0.5\linewidth]{image8.png}
    \caption{Error page}
    \label{fig:placeholder}
\end{figure}
\end{frame}

%-------------------------------------------------------
\begin{frame}{Possible Mitigations}


\begin{itemize}
    \item \textbf{Disable DTD and external entities:} Configure your XML parser to disallow DOCTYPE declarations and external entity resolution.
    \item \textbf{Use safe/hardened libraries:} Prefer libraries designed to prevent XXE.
    \item \textbf{Set XML resolvers to null:} Prevent fetching of external resources during parsing.
    \item \textbf{Validate and whitelist XML inputs:} Only allow expected elements, attributes, and content.
    \item \textbf{Prefer simpler formats if possible:} Use JSON or other non-XML formats when DTDs/entities are not needed.
    \item \textbf{Limit parser resources:} Enforce limits on document size, depth, entity expansions, and CPU/time consumption.
    \item \textbf{Run parsers with least privilege:} Ensure the parsing process cannot read sensitive files or perform privileged network calls.
   
\end{itemize}
\end{frame}
%-------------------------------------------------------




\end{document}

