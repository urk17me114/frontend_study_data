\documentclass{beamer}

\usepackage{graphicx}
\usepackage[absolute,overlay]{textpos}


\usetheme{Madrid}

\setbeamertemplate{navigation symbols}{}

\title[Requirements Analysis]{Requirements Analysis for the School System of Bobby's Fictional School}
\author{Arundas Mohandas \\ Ganga Sunil}
\institute{Bauhaus Universität - Weimar}
\date{16/04/2025}

% Remove default footline
\setbeamertemplate{footline}{}

% Add logo + page number automatically on all slides except title
\addtobeamertemplate{background}{
    \ifnum\value{framenumber}>1
        % Logo bottom-left
        \begin{textblock*}{2cm}(0.5cm,8.3cm)
            \includegraphics[height=0.7cm]{bauhaus university blue logo.jpeg}
        \end{textblock*}
        % Page number bottom-right
        \begin{textblock*}{2cm}(12cm,8.7cm)
            \usebeamercolor[fg]{page number in head/foot}\scriptsize \insertframenumber
        \end{textblock*}
    \fi
}{}

\begin{document}

% Title slide (no footer elements)
\begin{frame}
    \titlepage
\end{frame}

% Slide: Stakeholders
\begin{frame}
    \frametitle{Stakeholders}
    \begin{itemize}
        \item Students
        \item Teachers
        \item Parents
        \item Admin/Headmaster
    \end{itemize}
\end{frame}

% Slide: Requirements to Use Cases - Traceability Matrix 
\begin{frame}[fragile]
    \frametitle{Requirements to Use Cases - Traceability Matrix}
    \scriptsize
    \begin{tabular}{|l|p{6.3cm}|l|}
        \hline
        \textbf{Requirement ID} & \textbf{Requirement Description} & \textbf{Use Case(s)} \\
        \hline
        REQ1  & View student profile                 & UC1       \\
        REQ2  & Edit student profile                 & UC1       \\
        REQ3  & View student attendance              & UC3       \\
        REQ4  & View student marks                   & UC5       \\
        REQ5  & Apply/view leave (student)           & UC2, UC3   \\
        REQ6  & View timetable (student)             & UC8       \\
        REQ7  & Student messaging teachers           & UC11       \\
        REQ8  & View class schedule                  & UC8       \\
        REQ9  & Teacher takes attendance             & UC4       \\
        REQ10 & Teacher assigns marks                & UC6      \\
        REQ11 & View student leave requests          & UC7       \\
        REQ12 & Manage attendance                    & UC4       \\
        REQ13 & View teacher schedule                & UC8      \\
        REQ14 & Request leave (teacher)              & UC7      \\
        REQ15 & Message students                     & UC11       \\
        \hline
    \end{tabular}
\end{frame}

% Slide: Requirements to Use Cases - Traceability Matrix 
\begin{frame}[fragile]
    \frametitle{Requirements to Use Cases - Traceability Matrix }
    \scriptsize
    \begin{tabular}{|l|p{6.3cm}|l|}
        \hline
        \textbf{Requirement ID} & \textbf{Requirement Description} & \textbf{Use Case(s)} \\
        \hline
        REQ16 & Edit marks before publishing         & UC6      \\
        REQ17 & Parent views attendance              & UC3       \\
        REQ18 & Parent views marks                   & UC5       \\
        REQ19 & Parent views timetable               & UC8       \\
        REQ20 & View teacher info to students/parents & UC10       \\
        REQ21 & View announcements                   & UC12      \\
        REQ22 & Generate attendance reports          & UC14       \\
        REQ23 & Manage timetable                     & UC9       \\
        REQ24 & Edit student information in the system & UC16       \\
        REQ25 & Add/update holidays                  & UC13      \\
        REQ26 & Teacher communication                & UC11      \\
        REQ27 & Parent-teacher messaging             & UC11      \\
        REQ28 & Grade report of students             & UC14      \\
        REQ29 & Teachers can assign/modify assignments & UC15      \\
        REQ30 & Publish marks                        & UC7      \\
        REQ31 & Student registration                 & UC16      \\
        \hline
    \end{tabular}
\end{frame}

% Slide: Information Classification
\begin{frame}
    \frametitle{Information Classification}
    \scriptsize
    \vspace{-1cm}
    \textbf{Public Information:}
    \begin{itemize}
        \item Class Timetables
        \item School-wide Announcements
        \item Teacher Contact Information (limited: email, qualification, experience)
        \item General School Calendar
    \end{itemize}
    
    \textbf{Private Information:}
    \begin{itemize}
        \item Student Personal Details (address, date of birth, etc.)
        \item Marks and Grades
        \item Attendance Records
        \item Leave Applications
        \item Communication Messages (between stakeholders)
        \item Teacher-to-Teacher or Admin Notes
    \end{itemize}
    
    \textbf{Access Control:}
    \begin{itemize}
        \item \textbf{Students/Parents:} View own data only.
        \item \textbf{Teachers:} Access data related to assigned classes and students.
        \item \textbf{Admin/Headmaster's Office:} Full access to all school data.
    \end{itemize}
\end{frame}

% Slide: Use Case Descriptions Title
\begin{frame}
    \frametitle{Use Case Descriptions}
    \centering
    %\textbf{Use Case UC1: Update/View Personal Info} \\
    %\vspace{1cm}
    \textit{Below are the descriptions for all relevant use cases in the system.}
\end{frame}

% Slide: Use Case UC1
\begin{frame}
    \frametitle{Use Case UC1: Update/View Personal Info}
    \scriptsize  % Use scriptsize for even smaller font
    \vspace{-1cm}  % Remove extra space at the top
    \textbf{Related Requirements:} REQ1, REQ2 \\
    \textbf{Initiating Actor:} Student / Teacher \\
    \textbf{Actor’s Goal:} To view or update their personal profile information. \\
    \textbf{Participating Actors:} Admin Office \\
    \textbf{Preconditions:}
    \begin{itemize}
        \item User must be authenticated.
        \item The system must have existing personal records.
    \end{itemize}
    \textbf{Postconditions:}
    \begin{itemize}
        \item The user's information is updated or displayed.
    \end{itemize}
    \textbf{Flow of Events for Main Success Scenario:}
    \begin{enumerate}
        \item Student/Teacher logs into the system.
        \item Navigates to the "Profile" section.
        \item System displays current profile data.
        \item User edits desired fields and clicks "Save".
        \item System validates input and updates the database.
        \item System notifies the Admin Office if a change is significant (e.g., contact email).
    \end{enumerate}
\end{frame}

% Slide: Use Case UC2
\begin{frame}
    \frametitle{Use Case UC2: Apply for Leave}
    \scriptsize  % Use scriptsize for even smaller font
    \vspace{-1cm}  % Remove extra space at the top
    \textbf{Related Requirements:} REQ5 \\
    \textbf{Initiating Actor:} Student \\
    \textbf{Actor’s Goal:} To apply for individual leave. \\
    \textbf{Participating Actors:} Teacher \\
    \textbf{Preconditions:}
    \begin{itemize}
        \item Student must be logged in.
        \item Leave form must be filled.
    \end{itemize}
    \textbf{Postconditions:}
    \begin{itemize}
        \item Leave request is sent to the class teacher for review.
    \end{itemize}
    \textbf{Flow of Events for Main Success Scenario:}
    \begin{enumerate}
        \item Student accesses "Leave Application" form.
        \item Fills in date range and reason.
        \item Clicks “Submit”.
        \item System records the request.
        \item System forwards the request to the class teacher.
        \item Teacher can view and approve/reject the leave.
    \end{enumerate}
\end{frame}

% Slide: Use Case UC3
\begin{frame}
    \frametitle{Use Case UC3: View Attendance}
    \scriptsize  % Use scriptsize for even smaller font
    \vspace{-1cm}  % Adjusted space at the top
    \textbf{Related Requirements:} REQ3, REQ17 \\
    \textbf{Initiating Actor:} Student / Parent / Admin \\
    \textbf{Actor’s Goal:} To see attendance history or overview. \\
    \textbf{Participating Actors:} Teacher \\
    \textbf{Preconditions:}
    \begin{itemize}
        \item User must be authenticated.
        \item Attendance data must exist.
    \end{itemize}
    \textbf{Postconditions:}
    \begin{itemize}
        \item Attendance data is shown.
    \end{itemize}
    \textbf{Flow of Events for Main Success Scenario:}
    \begin{enumerate}
        \item Actor selects the "Attendance" option.
        \item System retrieves attendance data.
        \item Displays attendance in tabular or visual form.
    \end{enumerate}
\end{frame}

% Slide: Use Case UC4
\begin{frame}
    \frametitle{Use Case UC4: Manage Attendance}
    \scriptsize  % Use scriptsize for even smaller font
    \vspace{-1cm}  % Adjusted space at the top
    \textbf{Related Requirements:} REQ9, REQ12 \\
    \textbf{Initiating Actor:} Teacher \\
    \textbf{Actor’s Goal:} To mark or update attendance. \\
    \textbf{Participating Actors:} Admin Office \\
    \textbf{Preconditions:}
    \begin{itemize}
        \item Class is assigned to teacher.
        \item Current date/time is within valid school hours.
    \end{itemize}
    \textbf{Postconditions:}
    \begin{itemize}
        \item Attendance is updated in the system.
    \end{itemize}
    \textbf{Flow of Events for Main Success Scenario:}
    \begin{enumerate}
        \item Teacher opens class dashboard.
        \item Clicks on “Mark Attendance”.
        \item Selects present/absent/tardy for each student.
        \item Clicks “Submit”.
        \item System saves attendance record.
    \end{enumerate}
\end{frame}

% Slide: Use Case UC5
\begin{frame}
    \frametitle{Use Case UC5: View Marks}
    \scriptsize  % Use scriptsize for even smaller font
    \vspace{-1cm}  % Adjusted space at the top
    \textbf{Related Requirements:} REQ4, REQ18 \\
    \textbf{Initiating Actor:} Student / Parent \\
    \textbf{Actor’s Goal:} To check academic performance. \\
    \textbf{Participating Actors:} Teacher \\
    \textbf{Preconditions:}
    \begin{itemize}
        \item Marks must be uploaded and approved.
    \end{itemize}
    \textbf{Postconditions:}
    \begin{itemize}
        \item Marks are visible.
    \end{itemize}
    \textbf{Flow of Events for Main Success Scenario:}
    \begin{enumerate}
        \item Student/Parent selects the "Marks" section.
        \item System shows a list of subjects and corresponding marks.
        \item View is read-only.
    \end{enumerate}
\end{frame}

% Slide: Use Case UC6
\begin{frame}
    \frametitle{Use Case UC6: Upload/Edit Marks}
    \scriptsize  % Use scriptsize for even smaller font
    \vspace{-1cm}  % Adjusted space at the top
    \textbf{Related Requirements:} REQ10, REQ16 \\
    \textbf{Initiating Actor:} Teacher \\
    \textbf{Actor’s Goal:} To enter or update marks. \\
    \textbf{Participating Actors:} Admin Office \\
    \textbf{Preconditions:}
    \begin{itemize}
        \item Subject is assigned to the teacher.
    \end{itemize}
    \textbf{Postconditions:}
    \begin{itemize}
        \item Marks are recorded and pending approval.
    \end{itemize}
    \textbf{Flow of Events for Main Success Scenario:}
    \begin{enumerate}
        \item Teacher navigates to the “Upload Marks” section.
        \item Enters marks for each student.
        \item Clicks “Save”.
        \item System stores marks as “Pending”.
        \item Notifies Admin Office for approval.
    \end{enumerate}
\end{frame}

% Slide: Use Case UC7
\begin{frame}
    \frametitle{Use Case UC7: Approve Marks or Leave}
    \scriptsize  % Use scriptsize for even smaller font
    \vspace{-1cm}  % Adjusted space at the top
    \textbf{Related Requirements:} REQ11, REQ14, REQ30 \\
    \textbf{Initiating Actor:} Admin Office \\
    \textbf{Actor’s Goal:} To validate submissions from teachers or students. \\
    \textbf{Participating Actors:} None \\
    \textbf{Preconditions:}
    \begin{itemize}
        \item Requests must exist in the pending state.
    \end{itemize}
    \textbf{Postconditions:}
    \begin{itemize}
        \item Data is published or rejected.
    \end{itemize}
    \textbf{Flow of Events for Main Success Scenario:}
    \begin{enumerate}
        \item Admin reviews pending marks or leave requests.
        \item Accepts or rejects entries.
        \item System updates the record status accordingly.
    \end{enumerate}
\end{frame}

% Slide: Use Case UC8
\begin{frame}
    \frametitle{Use Case UC8: View Timetable}
    \scriptsize  % Use scriptsize for even smaller font
    \vspace{-1cm}  % Adjusted space at the top
    \textbf{Related Requirements:} REQ6, REQ8, REQ13, REQ19 \\
    \textbf{Initiating Actor:} Student / Parent / Teacher \\
    \textbf{Actor’s Goal:} To view class or personal timetable. \\
    \textbf{Participating Actors:} Admin Office \\
    \textbf{Preconditions:}
    \begin{itemize}
        \item Timetable must exist.
    \end{itemize}
    \textbf{Postconditions:}
    \begin{itemize}
        \item View-only access to the timetable.
    \end{itemize}
    \textbf{Flow of Events for Main Success Scenario:}
    \begin{enumerate}
        \item User selects “Timetable”.
        \item System filters relevant schedule based on role.
        \item Displays timetable.
    \end{enumerate}
\end{frame}

% Slide: Use Case UC9
\begin{frame}
    \frametitle{Use Case UC9: Create/Edit Timetable}
    \scriptsize  % Use scriptsize for even smaller font
    \vspace{-1cm}  % Adjusted space at the top
    \textbf{Related Requirements:} REQ23 \\
    \textbf{Initiating Actor:} Admin Office \\
    \textbf{Actor’s Goal:} To manage school-wide schedules. \\
    \textbf{Participating Actors:} Teachers \\
    \textbf{Preconditions:}
    \begin{itemize}
        \item Subject-teacher mappings must be available.
    \end{itemize}
    \textbf{Postconditions:}
    \begin{itemize}
        \item Updated timetable is active.
    \end{itemize}
    \textbf{Flow of Events for Main Success Scenario:}
    \begin{enumerate}
        \item Admin opens the “Schedule Manager”.
        \item Creates or modifies class slots.
        \item Assigns teachers and rooms.
        \item Clicks “Save”.
        \item Timetable is propagated to all stakeholders.
    \end{enumerate}
\end{frame}

% Slide: Use Case UC10
\begin{frame}
    \frametitle{Use Case UC10: View Teacher Info}
    \scriptsize  % Use scriptsize for even smaller font
    \vspace{-1cm}  % Adjusted space at the top
    \textbf{Related Requirements:} REQ20 \\
    \textbf{Initiating Actor:} Student / Parent \\
    \textbf{Actor’s Goal:} To see limited profile of teachers. \\
    \textbf{Participating Actors:} Teacher \\
    \textbf{Preconditions:}
    \begin{itemize}
        \item Teacher must be linked to a subject/class.
    \end{itemize}
    \textbf{Postconditions:}
    \begin{itemize}
        \item Contact and basic profile is visible.
    \end{itemize}
    \textbf{Flow of Events for Main Success Scenario:}
    \begin{enumerate}
        \item Actor selects a teacher from the class view.
        \item System displays limited contact data (email, qualification).
    \end{enumerate}
\end{frame}

% Slide: Use Case UC11
\begin{frame}
    \frametitle{Use Case UC11: Communicate with Stakeholders}
    \scriptsize  % Use scriptsize for even smaller font
    \vspace{-1cm}  % Adjusted space at the top
    \textbf{Related Requirements:} REQ7, REQ15, REQ26, REQ27 \\
    \textbf{Initiating Actor:} Student / Teacher / Parent \\
    \textbf{Actor’s Goal:} To send or receive messages. \\
    \textbf{Participating Actors:} Respective receivers \\
    \textbf{Preconditions:}
    \begin{itemize}
        \item Messaging system is active.
    \end{itemize}
    \textbf{Postconditions:}
    \begin{itemize}
        \item Message is stored and delivered.
    \end{itemize}
    \textbf{Flow of Events for Main Success Scenario:}
    \begin{enumerate}
        \item Actor opens “Messages”.
        \item Selects recipient (role-based filter).
        \item Types message and clicks “Send”.
        \item System notifies recipient.
    \end{enumerate}
\end{frame}

% Slide: Use Case UC12
\begin{frame}
    \frametitle{Use Case UC12: View Notifications}
    \scriptsize  % Use scriptsize for even smaller font
    \vspace{-1cm}  % Adjusted space at the top
    \textbf{Related Requirements:} REQ21 \\
    \textbf{Initiating Actor:} Student / Teacher / Parent \\
    \textbf{Actor’s Goal:} To stay updated with school news. \\
    \textbf{Participating Actors:} Admin \\
    \textbf{Preconditions:}
    \begin{itemize}
        \item Notifications must exist.
    \end{itemize}
    \textbf{Postconditions:}
    \begin{itemize}
        \item User is notified.
    \end{itemize}
    \textbf{Flow of Events for Main Success Scenario:}
    \begin{enumerate}
        \item User opens “Notifications” section.
        \item System lists all available updates.
        \item User reads notifications.
    \end{enumerate}
\end{frame}

% Slide: Use Case UC13
\begin{frame}
    \frametitle{Use Case UC13: Manage Holidays and Leaves}
    \scriptsize  % Use scriptsize for even smaller font
    \vspace{-1cm}  % Adjusted space at the top
    \textbf{Related Requirements:} REQ25 \\
    \textbf{Initiating Actor:} Admin \\
    \textbf{Actor’s Goal:} To add, edit, or delete holidays or leaves. \\
    \textbf{Participating Actors:} None \\
    \textbf{Preconditions:}
    \begin{itemize}
        \item User must have admin privileges.
    \end{itemize}
    \textbf{Postconditions:}
    \begin{itemize}
        \item Holiday/leave details are updated.
    \end{itemize}
    \textbf{Flow of Events for Main Success Scenario:}
    \begin{enumerate}
        \item Admin opens “Holiday Manager”.
        \item Admin adds or edits holiday details.
        \item Clicks “Save”.
        \item System updates the holiday data.
    \end{enumerate}
\end{frame}

% Slide: Use Case UC14
\begin{frame}
    \frametitle{Use Case UC14: Generate Reports}
    \scriptsize  % Use scriptsize for even smaller font
    \vspace{-1cm}  % Adjusted space at the top
    \textbf{Related Requirements:} REQ22, REQ28 \\
    \textbf{Initiating Actor:} Admin / Teacher \\
    \textbf{Actor’s Goal:} To generate performance or attendance reports. \\
    \textbf{Participating Actors:} None \\
    \textbf{Preconditions:}
    \begin{itemize}
        \item Data must be available in the system.
    \end{itemize}
    \textbf{Postconditions:}
    \begin{itemize}
        \item Reports are generated and available for download.
    \end{itemize}
    \textbf{Flow of Events for Main Success Scenario:}
    \begin{enumerate}
        \item User selects “Generate Report”.
        \item System prompts for report type (attendance, grades, etc.).
        \item System generates and presents the report.
    \end{enumerate}
\end{frame}

% Slide: Use Case UC15
\begin{frame}
    \frametitle{Use Case UC15: Manage Assignments}
    \scriptsize  % Use scriptsize for even smaller font
    \vspace{-1cm}  % Adjusted space at the top
    \textbf{Related Requirements:} REQ29 \\
    \textbf{Initiating Actor:} Teacher \\
    \textbf{Actor’s Goal:} To assign, modify or grade assignments. \\
    \textbf{Participating Actors:} Student \\
    \textbf{Preconditions:}
    \begin{itemize}
        \item Assignment must be created by the teacher.
    \end{itemize}
    \textbf{Postconditions:}
    \begin{itemize}
        \item Assignment details are available.
    \end{itemize}
    \textbf{Flow of Events for Main Success Scenario:}
    \begin{enumerate}
        \item Teacher creates or modifies assignments.
        \item Assignments are visible to students.
        \item Teacher grades assignments.
    \end{enumerate}
\end{frame}

% Slide: Use Case UC16
\begin{frame}
    \frametitle{Use Case UC16: Register/Update Student Info}
    \scriptsize  % Use scriptsize for even smaller font
    \vspace{-1cm}  % Adjusted space at the top
    \textbf{Related Requirements:} REQ31 \\
    \textbf{Initiating Actor:} Admin \\
    \textbf{Actor’s Goal:} To manage student registrations and updates. \\
    \textbf{Participating Actors:} None \\
    \textbf{Preconditions:}
    \begin{itemize}
        \item Admin has the required privileges.
    \end{itemize}
    \textbf{Postconditions:}
    \begin{itemize}
        \item Student info is registered/updated in the system.
    \end{itemize}
    \textbf{Flow of Events for Main Success Scenario:}
    \begin{enumerate}
        \item Admin accesses “Student Management”.
        \item Admin adds or updates student details.
        \item Clicks “Save”.
        \item System saves the data and confirms the action.
    \end{enumerate}
\end{frame}




% Slide: Technology Stack
\begin{frame}
    \frametitle{Technology Stack}
    \begin{itemize}
        \item \textbf{Backend:} Django
        \item \textbf{Frontend:} HTML, CSS, JavaScript
        \item \textbf{Database:} MySQL
        \item \textbf{Version Control:} GitHub
    \end{itemize}
\end{frame}

\end{document}
